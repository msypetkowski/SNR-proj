\documentclass[a4paper]{article}

\usepackage[a4paper,  margin=1.0in]{geometry}

\usepackage{graphicx}
\usepackage{float}
\usepackage{hyperref}


\usepackage[utf8]{inputenc}
\begin{document}


\title{SNR classes project - birds species recognition using deep neural networks}

\author{Michał Sypetkowski, Marcin Lew, Filip Smurawa}
\maketitle

\section{Data}
Training set is divided with ratio 0.1 (for testset)
so that number of examples of each class is equal in both training and test set.
In the end we got 300 examples for test set and 2700 raw examples for training set.

TODO: more

\section{HOG features}
TODO: mention used parameters, show visualization on an example


\section{Data augmentation}
Augmented examples for example shown on figure \ref{fig:aug1}
are shown on figure \ref{fig:aug2}.

TODO: describe data augmentation

\begin{figure}[h]
    \caption[]{Not augmented example}
    \centering
    \includegraphics[page=2,width=0.1\textwidth]{aug1.png}
    \label{fig:aug1}
\end{figure}

\begin{figure}[h]
    \caption[]{Augmented examples}
    \centering
    \includegraphics[page=2,width=1.0\textwidth]{aug2.png}
    \label{fig:aug2}
\end{figure}

\section{Model architecture}

TODO: experiment more and describe selected architecture

\section{Results}
Results on testset are shown on figure \ref{fig:eval}.

\begin{figure}[h]
    \caption[]{Results on testset (blue - good answers)}
    \centering
    \includegraphics[page=2,width=1.0\textwidth]{eval.png}
    \label{fig:eval}
\end{figure}

\section{What we are planning to do next}
Use deep convolutional network and feed whole images
(not extracted features e.g. with HOG).
With our implementation, such experiment requires only a few very simple modifications.


\end{document}
